\documentclass[12pt, xcolor={svgnames,table}]{beamer}
%\documentclass[20pt,handout]{beamer}
\usetheme{Darmstadt}
\usepackage{graphicx}
%\usepackage[german]{babel}
\usepackage{ngerman}
\usepackage[T1]{fontenc}
\usepackage[utf8]{inputenc}
\usepackage{tikz}
\setbeamertemplate{footline}[frame number]

\newcommand{\cc}[1]{\includegraphics[height=4mm]{img/#1.png}\hspace{1mm}}
\usepackage{ifthen}
\newcommand{\license}[2][]{\\#2\ifthenelse{\equal{#1}{}}{}{\\\scriptsize\url{#1}}}
\usepackage{textcomp}
\usepackage{hyperref}

\pgfdeclareimage[height=.6cm]{c3d2logo}{./img/c3d2.pdf}

\pgfdeclarelayer{foreground}
\pgfsetlayers{main,foreground}
\logo{\pgfputat{\pgfxy(-1,0)}{\pgfbox[center,base]{\pgfuseimage{c3d2logo}}}}

\title{Digitale Selbstverteidigung}
\author{\small Marius \\\large Chaos Computer Club Dresden}
\date{22.06.2016}

\begin{document}
\maketitle

\section{Einleitung}
\subsection{}

% ccc/ccc_bundesweit.tex
% ccc/ccc_lokal.tex
% motivation/stasi_vs_nsa.tex

\section{Internet}
\subsection{}

% internet/kommunikationsmodelle.tex
% internet/server_kabel_router.tex

\section{Gerätesicherheit}
\subsection{}

% geraetesicherheit/unerwuenschte_funktionalitaet.tex
% geraetesicherheit/schutzmoeglichkeiten.tex
% geraetesicherheit/app_permissions.tex
% geraetesicherheit/foss_programme_apps.tex

\section{Verschlüsselung}
\subsection{}

% verschluesselung/ssl.tex
% verschluesselung/https_everywhere.tex
% verschluesselung/e2e.tex
% verschluesselung/messenger_vergleich.tex

\section{Metadaten}
\subsection{}

% metadaten/vds.tex
% metadaten/location_heatmap.tex
% metadaten/zeitstempel.tex
% metadaten/antitracking.tex
% metadaten/alternative_dienste.tex
% metadaten/nextcloud.tex

\section{Soziale Netzwerke}
\subsection{}

% soziale_netzwerke/umfrage.tex
% soziale_netzwerke/geschaeftsmodelleraten.tex
% soziale_netzwerke/umgangsformen.tex

\section{Verhalten}
\subsection{}

% verhalten/datensparsamkeit.tex
% verhalten/passwoerter.tex

\section{Fazit}
\subsection{}

\begin{frame}
  \frametitle{Fazit}
  \begin{center}
    \begin{itemize}
      \item Wegverschlüsselung nutzen (auf HTTPS achten, Plugin: HTTPS Everywhere)
      \item Ende-zu-Ende-Verschlüsselung nutzen (Email: GPG, Messenger: Signal)
      \item Metadaten vermeiden (z.B. Alternative Dienste, Plugins: Privacy Badger oder Disconnect)
      \item Soziale Netzwerke bewusst nutzen, auf Umgangsformen achten
      \item Wichtig: Datenvermeidung und sichere Passwörter
    \end{itemize}

    \vspace{5mm}
    \href{https://github.com/cms/2016_05_04_kleinzschachwitz/}{Folien}: \href{https://creativecommons.org/licenses/by-sa/4.0/}{\cc{by-sa}} Chaos Computer Club Dresden \\
    \vspace{3mm}
    CMS Dresden: schule@c3d2.de
  \end{center}
\end{frame}

\end{document}
