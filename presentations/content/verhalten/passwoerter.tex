\begin{frame}
    \frametitle{Passwörter}
    \begin{itemize}
        \item<2-> Keine einfachen Wörter
        \item<3-> Groß-, Kleinbuchstaben, Ziffern, Sonderzeichen
        \item<4-> Beispiele:
            \begin{itemize}
                \item<5-> dragon
                \item<6-> (nCuAj.§Tsm!f
                \item<7-> IchLiebeDich
                \item<8-> .§)=/)=`
                \item<9-> qwerty
                \item<10-> Mks?o/.u,1Psw!
            \end{itemize}
        \item<12-> Verschiedene Passwörter nutzen!
        \item<13-> Passwort-Manager verwenden \\ (z.B. Keepass, Password Safe)
    \end{itemize}
\end{frame}

\note{Wichtiger Punkt, da dies selbst bei Erwachsenen oft noch nicht angekommen ist. Passwörter sollten MINDESTENS 8, besser 10 oder mehr Zeichen lang sein und nicht nur aus Kleinbuchstaben oder einem Wort bestehen. Das letzte Passwort ist sicher und einfach zu merken, weil es einem Satz folgt: ``Man kann sich fragen ob durch Punkt und Komma ein Passwort sicherer sird!''. Einen guten Effekt hat es, die Seite \url{https://howsecureismypassword.net/} zu zeigen und ein paar einfache Passwörter und zuletzt das lange gute einzugeben.}
