\begin{frame}
    \frametitle{Vertrauenswürdige Software?}
    \begin{center}\Large
        \enquote{Einer Software, die nicht quelloffen ist, kann man nicht vertrauen}
    \end{center}
\end{frame}

\begin{frame}
	\frametitle{Freie Software: Vier Freiheiten!}

	\begin{enumerate}
		\setcounter{enumi}{-1}
		\item das Programm auszuführen wie man möchte, für jeden Zweck
		\item die Funktionsweise des Programms zu untersuchen und eigenen Datenverarbeitungbedürfnissen anzupassen 
		\item das Programm zu redistribuieren und damit seinen Mitmenschen zu helfen
		\item das Programm zu verbessern und diese Verbesserungen der Öffentlichkeit freizugeben, damit alle davon profitieren
	\end{enumerate}
	
	Quelle: GNU Project (\url{https://www.gnu.org/philosophy/free-sw.de.html})
\end{frame}

\begin{frame}{Das GNU Projekt}
  \begin{columns}
    \column{6cm}

    \begin{itemize}
      \item Begonnen von Richard Stallman im Jahr 1984 
      \item Gründung der Free Software Foundation im Jahr 1985 
    \end{itemize}

    \column{7cm}

    \begin{center}
      \includegraphics[width=4.5cm]{img/stallman}
    \par\end{center}

    \begin{center}
      \includegraphics[width=5cm]{img/logo-fsf}
    \par\end{center}
  \end{columns}
\end{frame}

\begin{frame}
  \frametitle{Freie Software auf Computern}
    \begin{columns}
        \begin{column}{5cm}
            \begin{center}
                \includegraphics[height=0.2\textheight]{img/firefox.png} \\
                Firefox \\
                \vspace{0.1\textheight}
                \includegraphics[height=0.2\textheight]{img/libreoffice.jpg}\\
                LibreOffice
            \end{center}
        \end{column}
        \begin{column}{5cm}
            \begin{center}
                \includegraphics[height=0.2\textheight]{img/thunderbird.png} \\
                Thunderbird \\
                \vspace{0.1\textheight}
                \includegraphics[height=0.2\textheight]{img/vlc.png}\\
                VLC Media Player
            \end{center}
        \end{column}
    \end{columns}
\end{frame}

\begin{frame}{Freie Software auf dem Smartphone}
  \begin{columns}
    \column{6.5cm}

    \textbf{F-Droid}\\
    Android-Appstore für freie Software

    \vspace{0.5cm}

    \textbf{iOS Open Source Apps}\\
    \url{https://github.com/dkhamsing/open-source-ios-apps}

    \column{5cm}

    \begin{center}
      \includegraphics[width=2cm]{img/F-Droid_Logo_2}
    \par\end{center}
    \begin{center}
    \par\end{center}
  \end{columns}
\end{frame}
