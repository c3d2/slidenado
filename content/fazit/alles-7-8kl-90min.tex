\begin{frame}
  \frametitle{Fazit}
  \begin{center}
    \begin{itemize}
      \item Wegverschlüsselung nutzen (auf HTTPS achten, Plugin: HTTPS Everywhere)
      \item Ende-zu-Ende-Verschlüsselung nutzen (Email: GPG, Messenger: Signal, Conversations)
      \item Metadaten vermeiden (z.B. Alternative Dienste, Plugins: Privacy Badger oder Disconnect)
      \item Soziale Netzwerke bewusst nutzen, auf Umgangsformen achten
      \item Wichtig: Datenvermeidung und sichere Passwörter
    \end{itemize}

    \vspace{5mm}
    \href{https://github.com/cms/2016_05_04_kleinzschachwitz/}{Folien}: \href{https://creativecommons.org/licenses/by-sa/4.0/}{\cc{by-sa}} Chaos Computer Club Dresden \\
    \vspace{3mm}
    CMS Dresden: schule@c3d2.de
  \end{center}
\end{frame}

\note{Hier eine Übersicht zu allen Themenbereichen. Kurzfassung: HTTPS Everywhere nutzen um SSL/TLS immer zu nutzen, wenn es verfügbar ist. Ende zu Ende Verschlüsselung ist mittlerweile weitreichend verfügbar, jedoch sollte man einen open source messenger verwenden. Signal oder ein Jabber-Client mit OMEMO sind gute Optionen. Metadaten kann man vermeiden in dem man datenschutzfreundliche Dienste (z.B. Open Street Map, Startpage, etc.) nutzt und gegen Tracking ein Plugin installiert (Privacy Badger, Disconnect). Die beiden letzten Punkte sind selbsterklärend.}
