\begin{frame}
	\frametitle{Chaos Computer Club}

	\begin{center}
	\includegraphics[height=0.2\textheight]{../../img/chaosknoten.png}
	\end{center}
	\begin{itemize}
		\item<1-> Verein wurde 1981 gegründet (\url{https://ccc.de})
		\item<2-> auf dem jährlichen Kongress > 15.000 Besucher
		\item<3-> Technologie zum gesellschaftlichen Nutzen (und nicht ihrem Schaden)
		\item<4-> Betreibt u.a. Öffentlichkeitsarbeit und Politikberatung
	\end{itemize}
\end{frame}
\note{Der CCC ist eine lose Vereinigung von Hackern, die sich dafür einsetzen, Technik zum gesellschaftlichen Nutzen einzusetzen und nicht zu ihrem Schaden. Es wird daher auf unterschiedlichsten Ebenen, von politisch bis technisch, darauf hingewirkt, Technik nicht als magisch und unveränderlich wahrzunehmen sondern als etwas, das man selbst gestalten und verändern kann. Weiterhin setzt sich der CCC gegen Zensur und Überwachung ein.}

\begin{frame}
  \frametitle{Chaos Computer Club}
  \begin{figure}
    \includegraphics[height=0.7\textheight]{../../img/fingerabdruck.jpg}
  \end{figure}
\end{frame}
\note{Dies ist die erste von zwei Begebenheiten von der man den Chaos Computer Club, z.B. aus der Presse, kennen könnte. Üblicherweise fangen wir an zu fragen, was das abgebildete ist. Meist kommt als Antwort nur: ein Fingerabdruck. Tatsächlich ist es der Fingerabdruck von Herrn Schäuble, damals noch Innenminister. Der Fingerabdruck wurde bei einer Konferenz von einem Glas von Herrn Schäuble genommen und als Latexaufkleber für den Finger in der Datenschleuder, dem Magazin des CCC, verteilt. Es gibt einige bekannte Fälle in denen Mitglieder des CCC den Fingerabdruck von Herrn Schäuble im Reisepass haben, um damit zu zeigen, dass Fingerabdrücke kein geeignetes Identifikationsmerkmal darstellen und sich gegen die Abgabe der Fingerabdrücke zu widersetzen.}

\begin{frame}
  \frametitle{Chaos Computer Club}
  \begin{figure}
    \includegraphics[height=0.7\textheight]{../../img/trojaner.png}
  \end{figure}
\end{frame}
\note{Der Staatstrojaner ist die zweite bekannte Begebenheit. Dieser ist eine Schadsoftware die üblicherweise durch den Zoll auf den Computern von Verdächtigen auf einer Liste installiert wurde um später übers Internet erneut Zugriff auf das Gerät zu haben. Der CCC hat hier gezeigt, dass der Trojaner sehr viel mehr konnte als er verfassungsgemäß dürfte und hat damit den weiteren Einsatz unterbunden. Es ist bis heute umstritten, ob ein verfassungsgemäßer Einsatz von Trojanern überhaupt möglich ist.}
