\begin{frame}
    \frametitle{Metadaten im WWW}
    \begin{center} 
        \includegraphics<1>[width=0.7\textwidth]{img/lightbeam_1.png}
        \note{Ist es jemandem schonmal passiert, dass man sich etwas im Internet angeguckt hat und Tage und Wochen später noch Werbung auf ganz anderen Seiten für das gleiche gesehen hat? Das nennt sich Tracking und wird von vielen Firmen, u.a. Google und Facebook, gemacht um Leute auch auf anderen Webseiten verfolgen und ihre Aktivitäten mitschneiden zu können. }

        \includegraphics<2>[width=0.7\textwidth]{img/lightbeam_2.png}
        \note{Mit dem Addon Lightbeam für Firefox kann man sich anzeigen lassen, welche Tracker auf welchen Seiten eingebunden sind. An diesem Punkt bietet es sich an, das Plugin vorzuzeigen, es ggf. zu resetten und ein paar Websites aufzumachen, z.B. zeit.de, spon.de, google.de, facebook.com, web.de, gmx.de,... Besonders problematisch sind die Tracker (Dreiecke), die bei besonders vielen Websites (Kugeln) eingebunden sind.}
    \end{center}
\end{frame}

