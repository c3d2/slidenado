\begin{frame}
    \frametitle{Soziale Netzwerke}
    \begin{itemize}
        \item<2-> Das Internet vergisst nichts!
        \item<3-> Was sollte man (nicht) schreiben?
        \item<5-> Wer soll meine Infos (nicht) erhalten?
        \item<6-> Beeinträchtigt was ich schreibe andere?
    \end{itemize}
\end{frame}

\note{Zum ersten Punkt kann man sehr gut die Wayback-Machine zeigen: \url{https://archive.org/web/}. Hier kann man sich Websites bis zurück in die 90er Jahre angucken. Es zeigt sehr gut, dass alles was wir im Internet machen, einfach gepspeichert werden kann und auch wird. Alles was einmal im Internet ist, kann nicht mehr zurückgenommen werden. Deshalb ist es sehr wichtig, dass man sich VORHER Gedanken darüber macht, welche Informationen und Daten man so preisgibt, welche Fotos man hochlädt, aber auch ob ich dadurch die Privatsphäre von anderen mit beeinträchtige.}

\begin{frame}
  \frametitle{Umgangsformen}
  \begin{itemize}
    \item Wem sollte man was preisgeben (und was eher nicht)?
      \begin{itemize}
        \item<2-> "`Roadkill"' von Entertainment for the Braindead ist ein cooles Album
        \item<3-> Ich gehe am Donnerstag, den 07.06.2012 ins Kino
        \item<4-> Ich bin heut' echt gut drauf!
        \item<5-> Ute Meyer war mit Carolin Wittich und Frederik Ulm am Samstag Abend in der Sportsbar und ist danach mit Michael Müller nach Hause gefahren
        \item<6-> Meine Lehrerin ist voll doof!
      \end{itemize}
  \end{itemize}
\end{frame}

\note{Hier fragt man am besten die Schüler, welche Dinge ``ok'' sind und welche nicht. Meist ergeben sich ganz gute Diskussionen darüber à la ``Musik ist ok, aber wenn es z.B. eine rechtsextreme Band wäre...''. Das absolute No-Go sind die letzten beiden. Das vorletzte verletzt nicht nur die eigene Privatsphäre sondern auch die von anderen. Das letzte ist eine Beleidigung und gehört nicht ins Internet. Hier kann man kurz darüber diskutieren, wieso Leute im Internet manchmal Dinge schreiben, die sie im direkten Gespräch so nicht äußern würden. Beleidigungen, Sexismus, Rassismus etc.}
