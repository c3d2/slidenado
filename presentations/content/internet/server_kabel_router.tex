\begin{frame}
    \frametitle{Server im Rechenzentrum}
    \begin{center}
      \includegraphics[height=5cm]{img/data_center.jpg}
    \end{center}
\end{frame}

\note{Server stehen üblicherweise in großen Rechenzentren überall auf der Welt verteilt. Die abgebildeten Schränke heißen Racks und jeder Einschub ist ein Server, d.h. ein Computer, der keinen Bildschirm und keine Maus hat, dafür aber einen Netzwerkanschluss. Ein Server kann einen oder mehrere Dienste anbieten, z.B. eine abrufbare Website, einen Emailanbieter, einen Kommunikationsdienst für eine App etc. Große Dienste, wie Google oder Whatsapp sind üblicherweise auf viele Server in unterschiedlichen Rechenzentren verteilt.}

\begin{frame}
  \frametitle{Traceroute Le Monde}
    \begin{center}
      \includegraphics[height=5cm]{img/traceroute.png}
    \end{center}
\end{frame}

\note{In der Visualisierung eines 'traceroute', einer Verbindungsabfrage im Internet, sieht man manchmal verrückte Wege. Hier kann man beobachten, dass eine Anfrage von einem Computer in Deutschland an einen Server in Irland über die USA verläuft. Eine Route durchs Internet nimmt also nicht immer den geographisch kürzesten Weg, sondern orientiert sich oft eher an Auslastung der einzelnen Leitungen und auch damit verbundenen unterschiedlichen Preisen.}

\begin{frame}
  \frametitle{Traceroute Facebook}
    \begin{center}
      \includegraphics[height=5cm]{img/traceroute-facebook.png}
    \end{center}
\end{frame}
\note{Im Traceroute von Facebook sieht man, dass JEDE Anfrage an Facebook erstmal in die USA geht bevor sie dort feststellen, dass jemand aus Europa kommt und das dann auf die Server in Irland umleiten. Trotzdem hat der amerikanische Facebook-Zweig damit auch unsere Daten und nicht nur der datenschutzmäßig besser geschützte Teil in Europa.}

\begin{frame}
    \frametitle{Internetknoten (Router)}
    \begin{center}
      \includegraphics[height=5cm]{img/internet_cable_map.png}
    \end{center}
\end{frame}

\note{Das Versenden einer Anfrage (z.B. der Aufruf einer Website) übers Internet ist vergleichbar mit dem Verschicken eines Postpakets, man gibt es an der lokalen Post ab und dann wird es über kleinere und größere Paketverteilstationen - ggf. übers Meer - zu seinem Empfänger gebracht. Diese Verteilzentren nennen sich im Internet Router oder Switches und sind dafür da, dass die Internetpakete nach vielen Schritten vom Nutzer zum Server und im Falle einer Antwort auch wieder zurückkommen.}

\begin{frame}
    \frametitle{Internetknoten (DE-CIX in Frankfurt)}
    \begin{center}
      \includegraphics[height=5cm]{img/de_cix.jpg}
      \\{\small \href{https://de.wikipedia.org/wiki/DE-CIX\#/media/File:DE-CIX\_GERMANY\_-\_Switch\_Rack\_\%286218137120\%29.jpg}{Grafik}: \href{https://creativecommons.org/licenses/by-sa/2.0/}{\cc{by-sa} Stefan Funke}}
    \end{center}
\end{frame}

\note{Der abgebildete Verteilknoten ist der DE-CIX in Frankfurt, der Größte der Welt, der nicht aus einem sondern aus ganz vielen Routern besteht.}

\begin{frame}
    \frametitle{Verbindungskabel}
    \begin{center}
      \includegraphics[height=5cm]{img/seacable1.jpg}
    \end{center}
\end{frame}

\note{Hier sieht man das Verlegen eines Unterseekabels, welches dann z.B. zwei Kontinente miteinander verbindet}
