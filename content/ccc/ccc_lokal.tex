\begin{frame}
    \frametitle{Chaos Computer Club}
    \begin{center}
	\includegraphics[height=0.1\textheight]{../../img/c3d2_logo.png}
    \end{center}
    \begin{itemize}
      \item<1-> Chaos Computer Club Dresden (\url{https://c3d2.de})
      \item<2-> Datenspuren: Herbst 2016 (\url{https://datenspuren.de})
      \item<3-> Podcasts (\url{https://c3d2.de/radio.html})
      \item<4-> IT4Refugees
      \item<5-> Chaos macht Schule (\url{https://c3d2.de/schule.html})
    \end{itemize}
\end{frame}

\note{Der Dresdner CCC ist vorrangig ein Treffpunkt für alle, die die Werte des CCC teilen. Hier kann man sich treffen, austauschen, Projekte organisieren, Themenabende abhalten etc. Das größte gemeinsame Projekt in Dresden sind die Datenspuren, eine kostenlose Konferenz für jeden Bürger mit Vorträgen und Workshops zu den Themen Datenschutz, Datensicherheit und bewusster Umgang mit neuen Technologien. Der C3D2 (= CCC DD) macht weiterhin Radio im lokalen Pentaradio und Podcasts. Zwei nennenwerte Projekte des C3D2 sind IT4Refugees, die über Freifunk Internetzugang in Flüchtlingsunterkünfte bringen und unser Projekt, Chaos macht Schule, das Vorträge und Workshops in Schulen, aber auch an Unis und anderen Einrichtungen macht, um zu zeigen, wie man seine Daten und seine Privatsphäre im Internet schützen kann.}
