\begin{frame}
    \frametitle{Datensparsamkeit}
    \begin{itemize}
        \item<2-> Viele Daten zusammen ergeben Profile
        \item<3-> Verteilen der Daten, Nutzung mehrerer Dienste
        \item<4-> Werden die Daten gebraucht?
        \item<5-> Werden echte Daten gebraucht?
          \begin{itemize}
            \item<6-> Pseudonymität
            \item<7-> mailinator.com (Wegwerf-Email-Adresse)
            \item<8-> frank-geht-ran.de (Wegwerf-Telefonnummer)
            \item<9-> bugmenot.com (Fake Accounts)
          \end{itemize}
    \end{itemize}
\end{frame}

\note{Wichtigster Teil: Man braucht nicht immer seine Daten überall anzugeben. Und selbst wenn der Dienst einen die Daten nicht weglassen lässt, ist es total ok falsche anzugeben. Pseudonyme = falsche Namen bzw. Phantasienamen. Mit mailinator.com hat man für 10 Minuten eine Emailadresse, mit der man sich irgendwo registrieren kann und dort auf den Bestätigunglink klickt. Wenn man nur mal einen Website ``von innen'' sehen will, kann man auch auf Bugmenot gucken, ob schon jemand einen Fakeaccount angelegt hat.}
